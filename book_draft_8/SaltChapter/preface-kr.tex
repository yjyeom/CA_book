% 옮긴이 머리말은 나중에 포함시킨다.


\chapter*[옮긴이 머리말]{옮긴이 머리말}
\addcontentsline{toc}{chapter}{옮긴이 머리말}


%==============================

복소해석학에서는 실함수의 미적분학을 
복소수 공간으로 확장한 결과를 살펴보며,
역으로 복소수 공간에서 얻은 성질을 
실함수 공간으로 가져와서 유용하게 응용하는 방법을 학습한다. 
마치 친숙한 실수 공간에서 출발하여
복소수 공간으로 여행을 다녀오는 것으로 생각할 수 있다.
모든 여행의 끝은 출발점으로 돌아오는 것인데
여행 일정에 	볼 것을 너무 많이 넣으면  돌아오지 못할 수도 있다.
이 책의 특징은 매우 간결함이고, 우리는 복소수 여행에서
중요하고 필수적인 것에 집중하여
길을 잃지 않고 유익게 사용할 수 있는 기념품을 챙겨 빠르게 출발점으로 복귀하고자 한다.


복소해석학 과목에서 사용할 수 있는 좋은 교재가 많지만
대부분 한학기 과정으로는 조금 부담되는 분량이다. 
대학교과에서 다양한 응용분야를 다루어야 하고 
복수전공을 권장하는 변화에 따라 핵심적인 부분을 중심으로 과정을 간결하게 만들 필요가 있다,
이 책은 복소해석함수와 실변수 조화함수를 연결하는 것을
복소해석학의 주 목적으로 하여 간결하고 빠르게 이야기를 전개한다.
따라서 줄거리에서 벗어나는 개념들은 생략한 부분이 있는데
예를 들면, 다가함수 개념과 브렌치, 등각사상의 예제, 미탁-레플러 정리 등은
아쉽게도 다루지 않는다.
하지만, 과감한 생략이 끝까지 완주하는데 분명 도움이 될 것으로 보이며
생략된 부분은 강의하는 과정에서 필요하면 채우는데 어려움이 없을 것으로 예상된다.

이 책의 장점 중 하나는 모든 연습문제의 풀이가 있다는 것인데,
학습한 내용을 확인하는 문제도 있지만
줄거리를 전개하는데 필요한 부분을 연습문제로 돌린 경우도 있다.
따라서 학생들은 문제의 난이도에 넘 좌절하지 않길 바란다.
해석학의 공부에는 $\epsilon$-$\delta$를 이용한 엄밀한 극한의 정의가
필수적인데, 이 기법에 익숙하지 않은 경우, 정리의 증명이나 연습문제에서 
해당부분을 받아들이고 진행해도 전체 줄거리와 주요 결과를 이해하는데 무리가 없다.

모든 여행은 출발점으로 돌아오는 것이 목적이기 때문에
모든 독자들이 책의 끝부분에 성공적으로 도달하길 바라며,
필요시 다른 책으로 복소수의 다른 면모를 느껴보길 권한다.


%============================


\clearpage





