% !TEX root = ../notes_template.tex

\chapter{코시 적분 정리와 응용}

복소미분은 어느정도 익숙해졌으니 이제 적분으로 관심을 돌려보자.
이 장에서는 복소해석학에서 매우 중요한 다음 정리를 배울 예정이다.
\begin{center}
\fbox{
코시 적분 정리
}
\end{center}
``경로적분''을 정의하는 것으로 시작하여 나중에 코시 적분 정리를 증명할 예정이다.
왜 경로적분과 코시 적분 정리가 왜 그렇게 중요한지 의문을 가질 수 있다.
복소평면에서 적분의 중요성은 복소해석함수의 더 큰 이해로 이어지기 때문이다.
예를 들면, 복소해석함수는 무한번 미분가능하다는 본질적인 성질이 있다.
이 장에서 다음 주제들을 중심으로 공부해보기로 하자.
\begin{enumerate}
\item[(1)] 경로적분의 정의와 성질
\item[(2)] 경로적분의 기본정리
\item[(3)] 코시 적분 정리
\item[(4)] 코시 적분 정리의 응용
\begin{enumerate}
\item 부정적분의 존재성
\item 복소해석함수의 무한번 미분가능성
\item 리우비우 정리와 대수학의 기본정리
\item 모레라 정리
\end{enumerate}
\end{enumerate}

\section{경로적분의 정의}

일반적인 미적분에서 연속함수 $f: [a,b] \to \mathbb R$가 주어질 때
\begin{equation}\label{eq-3-1}
\int_a^b f(x)dx
\end{equation}
의 의미는 명확하다. 이제 이를  일반화하여 복소수까지 확장하고
주어진 복소수 $z$, $w$에 대하여
\[
\int_z^w f(\zeta)d\zeta
\]
에 의미를 부여하길 원한다고 하자.
$z$에서 $w$까지를 어떻게 해석해야 할까?

$\mathbb R$에서 $a<b$이면, 실수 $a$부터  실수 $b$까지
가는 경로는 한가지 뿐이다.  
따라서 실수의 경우는 단지
\begin{itemize}
\item[(1)] $a<b$이고,
\item[(2)] 연속함수 $f:[a,b] \to \mathbb R$
\end{itemize}
의 경우만 생각하면 충분하다.

하지만, $z$와 $w$가 복소평면 위의 점이면
그림 \ref{fig-3-1}과 같이 많은 경로에 대하여 적분을 생각할 수 있다.

\begin{figure}[!h]
\begin{center}
\includegraphics[width=0.8\textwidth]{./SaltChapter/fig-3-1}
\end{center}
\caption{$z$에서 $w$까지 어떤 경로로 가야할까?}
\label{fig-3-1}
\end{figure}

그러므로 복소수의 경우는 끝점 $z$와 $w$ 외에
$z$에서 $w$까지의 경로 $\gamma$도 지정하고,
실수의 경우를 나타낸 식 \eqref{eq-3-1}를
다음과 같이 복소수에 대한 표현으로 바꾸도록 한다.
\[
\int_\gamma f(z)dz.
\]

이 표현을 ``경로''적분이라 부르며
계산을 위해 다음을 결정할 필요가 있다.
\begin{itemize}
\item[(1)] 정의역  $D(\subset \mathbb C)$와 $z, w\in D$
\item[(2)] 연속함수 $f:D\to \mathbb C$
\item[(3)] $z$와 $w$를 잇는 {\bf 매끄러운} 경로 $\gamma: [a,b] \to D$
\end{itemize}

$z$와 $w$를 단순히 연결하는 경로가 아니라  {\bf 매끄러운} 경로가
필요하다는 사실에 주목하자.
여기서 ``매끄럽다''는 의미는 무엇일까?
경로 $\gamma : [a,b] \to D$는 연속함수임을 상기하자.
$\gamma$는 실수부와 허수부 $x,y: [a,b] \to \mathbb R$로 나누어  쓸 수 있다.
\[
\gamma(t) = x(t) + iy(t), \quad t\in [a,b].
\]
$x, y$가 연속미분가능하면
경로 $\gamma$가 {\bf 매끄럽다}고 한다.
예를 살펴보자.

\begin{salt_example} \label{example-3-1}
$\gamma : [0,1] \to \mathbb C$를 
$\gamma(t) = t(1+i)$ ($t\in[0,1]$)로 정의하자.
그러면 $\gamma$의 실수부와 허수부  $x,y: [a,b] \to \mathbb R$는
$x(t)=t$, $y(t)=t$, $t\in [0,1]$이 된다.
$x, y$가 $[0,1]$에서 연속미분가능이므로 $\gamma$는 매끄러운 곡선이다.
그림 \ref{fig-3-2}를 참고하라.
\begin{figure}[!h]
\begin{center}
\includegraphics[width=0.4\textwidth]{./SaltChapter/fig-3-2}
\end{center}
\caption{매끄러운 곡선 $\gamma$}
\label{fig-3-2}
\end{figure}
비슷한 방법으로 다음과 같이 주어진 두 경로  $\gamma_1, \gamma_2: [0,2\pi] \to \mathbb R$를
생각하자.
\[
\gamma_1(t) = \exp(it), \quad \gamma_2(t) = \exp(2it), \quad t\in [0,2\pi].
\]
그러면 이 경로들의 실수부와 허수부는
$\cos t$, $\sin t$, $\cos(2t)$, $\sin(2t)$이고
모두 연속미분가능하다.
따라서 $\gamma_1, \gamma_2$는 모두 매끄러운 경로이다.
그림 \ref{fig-3-3}을 보자.
\begin{figure}[!h]
\begin{center}
\includegraphics[width=0.6\textwidth]{./SaltChapter/fig-3-3}
\end{center}
\caption{매끄러운 곡선 $\gamma_1$과 $\gamma_2$}
\label{fig-3-3}
\end{figure}
두 곡선의 이미지 ($\gamma_1$과 $\gamma_2$의 치역)은 동일하다.
즉, 중심이 원점인 단위원이다.
\[
\{\gamma_1(t) \,:\, t\in[0,2\pi]\}  
= \{\gamma_2(t) \,:\, t\in[0,2\pi]\}  
= \{z\in\mathbb C \,:\, |z|=1 \}. 
\] 
그렇지만 $\gamma_1$과 $\gamma_2$는 다른 경로이다.
왜냐하면 함수로서 같지 않기 때문이다. 예를 들면
$\gamma_1(\pi) = -1 \ne 1 = \gamma_2(\pi)$.
\hfill $\diamondsuit$
\end{salt_example}

\begin{salt_remark} \label{remark-3-1}
경로 $\gamma: [a,b]\to \mathbb C$의 치역
\[
\{\gamma(t) \,:\, t\in[a,b] \}
\]
을 경로(또는 곡선) 자체라고 하는 것이 매우 일반적이며 편리하다. 
이 방식에서는 경로는 
복소평명에서의 원, 선분 구체적인 기하학적 개체가 되어 (함수라고 생각하는 것과 반대로),
쉽게 그려볼 수 있다.
이 방식에서는 {\bf 다른} 경로를 동일한 이미지로 볼 수 있어
모호함이 생긴다는 어려움이 있다.
\end{salt_remark}

경로적분의 정확한 정의는 다음과 같다.

\begin{salt_definition} \label{def-3-1}
다음이 주어질 때,
\begin{itemize}
\item[(1)] 정의역 $D$,
\item[(2)] 연속함수  $f:D\to\mathbb C$ (실수부와 허수부는 
$u,v: D\to\mathbb R$),
\item[(3)] 매끄러운 경로 $\gamma : [a,b]\to D$
(실수부와 허수부는 $x,y: [a,b] \to \mathbb R$),
\end{itemize}
경로적분을 다음과 같이 정의한다.
\begin{align} \label{eq-3-2}
\int_\gamma f(z)dz
&:= \int_a^b f(\gamma(t))\gamma'(t)dt \\
&:= \int_a^b \left( u(\gamma(t)) + iv(\gamma(t)) \right) \cdot
(x'(t) + iy'(t))dt \nonumber \\
&:= \int_a^b \left( u(\gamma(t)\cdot x'(t) - v(\gamma(t))\cdot y'(t) \right) \nonumber \\
&\quad +i \int_a^b \left(v(\gamma(t)\cdot x'(t) + u(\gamma(t))\cdot y'(t) \right) dt \nonumber.
\end{align}
여기서 마지막 두 적분은 우리에게 익숙한 실변수 연속함수의 리만적분이다.
\end{salt_definition}

다음과 같이 경로적분을 기하학적으로 해석할 수 있다.
\[
\gamma'(t) dt = x'(t)dt + iy'(t)dt
\]
이 항을 경로를 따라 국소적으로 변하는 증분으로 보자.
이 증분에 값 $f(\gamma(t))$ (국소적으로는 거의 상수이다)을 곱하고,
경로를 따라 더해나가면 결론적으로 적분값
\[
\int_a^b f(\gamma(t))\gamma'(t)dt
\]
에 도달하게 된다. 그림 \ref{fig-3-4}를 참고하라.

\begin{figure}[!h]
\begin{center}
\includegraphics[width=0.8\textwidth]{./SaltChapter/fig-3-4}
\end{center}
\caption{경로적분의 기하학적 의미}
\label{fig-3-4}
\end{figure}

\begin{salt_example} \label{example-3-2}
다음과 같이 주어진 조건에 대하여
\begin{itemize}
\item[(1)] $D=\mathbb C$,
\item[(2)] $\gamma$는 $\gamma(t)= t(1+i)$ ($t\in[0,1]$)로 정의된 매끄러운 경로,
\item[(3)] $f = (z\to \bar z)$,
\end{itemize}
\begin{align*}
\int_\gamma f(z)dz &= \int_0^1 \overline{t(1+i)}\cdot (1+i)dt \\
&= \int_0^1 t(1-i)\cdot(1+i)dt
= \int_0^1 t(1^2-i^2)dt = \int_0^1 t(1+1)dt \\
&= 2\int_0^1 t\,dt = 2\cdot \frac{t^2}2 \Big|_0^1 = 2\cdot \dfrac12 = 1.
\end{align*}
\end{salt_example}

\begin{salt_exercise} \label{ex-3-1}
세 경로 $\gamma_1, \gamma_2, \gamma_3: [0,2\pi] \to \mathbb C$가 
$t\in[0,2\pi]$에 대하여 다음과 같이 정의된다고 하자.
\begin{align*}
\gamma_1(t) &= \exp(it), \\
\gamma_2(t) &= \exp(2it), \\
\gamma_3(t) &= \exp(-it).
\end{align*}
경로의 이미지는 모두 같지만, 다음 세 경로적분은 모두 다른 값을 가짐을 보여라.
\[
\int_{\gamma_1} \dfrac1zdz, \quad
\int_{\gamma_2} \dfrac1zdz, \quad
\int_{\gamma_3} \dfrac1zdz.
\]
\end{salt_exercise}

\begin{salt_exercise} \label{ex-3-2}
$f$가 영역 $D$에서 복소해석함수이고, $\gamma:[0,1]\to D$가 매끄러운 경로라 하자.
모든 $t\in[0,1]$에 대하여 다음을 증명하라.
\[
\dfrac{d}{dt} f(\gamma(t)) = f'(\gamma(t))\cdot \gamma'(t). 
\]
\end{salt_exercise}

우리는 종종 일반적인 구간 $[a,b]$를 사용하지 않고
매끄러운 경로가 $[0,1]$에서 매개변수로 정의된 것으로 가정하기도 한다.
왜 이런 가정을 해도 되는지 이유를 설명해보자.

$\gamma:[a,b] \to \mathbb C$와  $\tilde\gamma:[a,b] \to \mathbb C$가
매끄러운 경로라고 하자.
연속미분가능한 함수 $\varphi:[c,d] \to [a,b]$가
$a=\varphi(c)$, $b=\varphi(d)$이고 모든 $t\in[c,d]$에 대하여
$\tilde\gamma(t) = \gamma(\varphi(t))$를 만족한다고 하자.
이러한 두 경로를 ``동치''라고 한다.
$\gamma(a) = \tilde\gamma(c)$부터 $\gamma(b) = \tilde\gamma(d)$까지
동일한 길을 따라 간다고 상상해보자. 단, 속도는 다를 수 있다.
그림 \ref{fig-3-5}를 보자. 
\begin{figure}[!h]
\begin{center}
\includegraphics[width=0.7\textwidth]{./SaltChapter/fig-3-5}
\end{center}
\caption{동치 경로}
\label{fig-3-5}
\end{figure}
이제 다음 결과를 보일 수 있다.

{\bf 동치 경로에 대한 적분결과는 동일하다:}
연쇄법칙에 의해 다음이 성립한다.
\begin{align*}
\int_{\tilde\gamma}f(z)dz
&= \int_c^d f(\tilde\gamma(t))\tilde\gamma'(t)dt
= \int_c^d f(\gamma(\varphi(t)))\gamma'(\varphi(t)) \varphi'(t)dt \\
&\stackrel{(\tau=\varphi(t))}=
\int_a^b f(\gamma(\tau))\gamma'(\tau)d\tau
= \int_{\gamma}f(z)dz.
\end{align*}
특히, 주어진 $\gamma:[a,b]\to\mathbb C$에 대하여
$\varphi: [0,1]\to [a,b]$를 다음과 같이 정의하자.
\[
\varphi(t) = (1-t)a + tb, \quad t\in [a,b].
\]
그러면 $\varphi$는 연속미분가능하고,
$\varphi(0)=a$, $\varphi(1)=b$이다.
따라서 $c:=0$, $d:=1$로 두고 위의 결과를 적용하면,
$\tilde\gamma : [0,1]\to\mathbb C$를
$\tilde\gamma = \gamma\circ\varphi$라 정의하여
다음을 얻는다.
\[
\int_{\tilde\gamma} f(z)dz = \int_\gamma f(z)dz.
\]
결론적으로, 경로적분과 관련하여
매끄러운 곡선은 $[0,1]$에서 매개화된 것으로 간주해도 일반성을 잃지 않는다.

{\bf 조각적으로 매끄러운 경로에 대한 경로적분:}
경로의 정의를 ``꺽인 점''을 갖는 경로까지 확장해보자.
점 $c_1, \ldots, c_n$가
\[
a<c_1 < \cdots <c_n <b
\]
를 만족하고 $\gamma$가 구간 $[a,c_1], [c_1,c_2], \ldots, [c_{n_1}, c_n], [c_n,b]$ 각각에서
연속미분가능할 때,
경로 $\gamma:[a,b]\to\mathbb C$가 {\bf 조각적으로 매끄러운 경로 또는 곡선}이라 한다.
이러한 경로에서의 적분은 다음과 같이 정의한다.
\begin{align*}
\int_\gamma f(z)dz
&:= \int_a^{c_1} f(\gamma(t))\gamma'(t)dt + \int_{c_1}^{c_2} f(\gamma(t))\gamma'(t)dt
+ \cdots \\
&\quad \quad + \int_{c_{n-1}}^{c_n} f(\gamma(t))\gamma'(t)dt 
+ \int_{c_n}^b f(\gamma(t))\gamma'(t)dt.
\end{align*}

\begin{salt_example} \label{example-3-3}
$0$부터 $1+i$까지의 경로 $\tilde\gamma$가 다음과 같이 정의된다고 하자.
\[
\tilde\gamma(t) = \begin{cases}
t, & t\in[0,1], \\
1+(t-1)i, t\in (1,2].
\end{cases}
\]
그림 \ref{fig-3-6}을 보자.
\begin{figure}[!h]
\begin{center}
\includegraphics[width=0.4\textwidth]{./SaltChapter/fig-3-6}
\end{center}
\caption{조각적으로 매끄러운 경로 $\tilde\gamma$}
\label{fig-3-6}
\end{figure}
그러면
\begin{align*}
\int_{\tilde\gamma} \bar z dz 
&= \int_0^1 \bar t \,1\,dt
+ \int_1^2 \overline{(1+(t-1)i)}\,i\, dt
= \int_0^t t\,dt + \int_1^2 (1-(t-1)i)i\,dt \\
&= \int_0^1 t\,dt + \int_1^2 (i+(t-1))dt \\
&= \dfrac12 + i + \dfrac{4-1}2 - 1 = 1+ i.
\end{align*}
\end{salt_example}

예제 \ref{example-3-2}와 \ref{example-3-3}에서 얻은 계산결과를 돌아보자.
피적분함수는 같고($z\to\bar z$로 복소해석함수는 아니다),
그림 \ref{fig-3-7}과 같이
동일한 양끝점 $0$과 $1+i$를 연결하는 두 경로 $\gamma$와 $\tilde\gamma$에 대하여
\begin{figure}[!h]
\begin{center}
\includegraphics[width=0.4\textwidth]{./SaltChapter/fig-3-7}
\end{center}
\caption{두 경로 $\gamma$와 $\tilde\gamma$}
\label{fig-3-7}
\end{figure}
다른 적분 결과를 얻었다.
\[
\int_\gamma \bar z dz = 1 \ne 1+i = \int_{\tilde\gamma}\bar z dz.
\]
따라서 복소해석함수가 아닌 피적분함수 $z\to\bar z$는 경로에 따라 적분결과가 다르다.
경로적분의 정의를 보면
선택한 길에 따라 계산된 경로적분의 값이 다를 것으로 기대되기 때문에
뜻밖의 결과가 아니다.
이 장에서 중요한 목표는 
점 $z$에서 $w$까지 연결하는 두 경로 사이의 영역에서 복소해석적인 함수에 대해서는
두 경로를 따라 적분한 결과는 동일함을 보이는 것이다.
이 결과는 코시 적분 정리라 불리는 복소해석학의 핵심 결과로 이어진다.

\begin{salt_example} \label{example-3-4}
앞의 예제 \ref{example-3-2}와   \ref{example-3-3}에서 정의한
경로 $\gamma$, $\tilde\gamma$를 생각하자.
이번에는 피적분함수로 복소해석함수가 아니었던 $z\mapsto\bar z$ 대신 
전해석함수 $z$를 사용하자. 그러면,
\begin{align*}
\int_\gamma z dz 
&= \int_0^1 (1+i)t(1+i)dt = \int_0^1 2it\,dt = i \text{ 이고,} \\
\int_{\tilde\gamma} z dz
&= \int_0^1 t\cdot1\, dt + \int_1^2 (1+(t-1)i)i\, dt \\
&= \int_0^1 t\, dt + \int_1^2 (i-(t-1))dt
= \dfrac 12 - \dfrac12 + i = i  \text{ 이므로}
\end{align*}
경로 $\gamma$와 경로 $\tilde\gamma$를 따른  경로적분값이 동일하다.
\hfill $\diamondsuit$
\end{salt_example}

\begin{salt_exercise} \label{ex-3-3}
원 $|z|=2$를 따라 반시계방향의 한바퀴 도는 경로로 다음 함수를 적분하라.
\begin{itemize}
\item[(1)] $z+\bar z$
\item[(2)] $z^2-2z+3$
\item[(3)] $xy$ ($z=x+iy$, $x,y\in\mathbb R$)
\end{itemize}
\end{salt_exercise}


\begin{salt_exercise} \label{ex-3-4}
다음 경로 $\gamma$를 따라 적분 $\dint_\gamma \Re(z)dz$를 계산하라.
\begin{itemize}
\item[(1)] $0$에서 $1+i$까지 직선으로 연결한 선분
\item[(2)] 중심이 $i$이고 반지름이 $1$인 원을 따라 $0$부터 $1+i$까지 연결한 원호
\item[(3)] 포물선 $y=x^2$위에서 $x=0$부터 $x=1$까지 $0$과 $1+i$를 연결한 곡선
\end{itemize}
\end{salt_exercise}

\subsection{하나의 중요한 적분계산}

여기서 간단하지만 매우 중요한 경로적분 하나를 계산할 것인데
이 적분은 앞으로 계속 반복하여 다시 돌아볼 예정이다.
하나의 규칙을 정하자: 이 책 전체를 통하여 특별히 언급하지 않으면
$z_0$를 중심으로 반지름이 $r>0$인 원을 따라 반시계방향으로 도는 경로
$C: [0, 2\pi] \to \mathbb C$는 $C(t) = z_0 + r\exp(it)$, $t\in[0,2\pi]$로 정의한다
(따라서 한바퀴만 돈다). 그림 \ref{fig-3-8}을 참고하라.
\begin{figure}[!h]
\begin{center}
\includegraphics[width=0.9\textwidth]{./SaltChapter/fig-3-8}
\end{center}
\caption{$z_0$를 중심으로 반지름이 $r$인 원을 반시계방향으로 도는 경로 $C$}
\label{fig-3-8}
\end{figure}

이제 적분 $\int_C (z-z_0)^n dz$ ($n\in\mathbb Z$)를 계산하자.
나중에 이 계산이 매우 유용한 것으로 입증될 것이다.

\begin{salt_theorem} \label{thm-3-1}
$C$를 중심이 $z_0$이고 반지름 $r>0$인 원을 반시계방향으로 도는 경로라 하자.
그러면,
\[
\int_C (z-z_0)^n dz = \begin{cases}
2\pi i, & n=-1,\\
0, & n\ne -1.
\end{cases}
\]
여기서 적분값은 $r$에 무관함을 알 수 있다.
\end{salt_theorem}

{\bf 증명}

경로는 $C(t) = z_0 + r\exp(it) = z_0 + r\cos t + it\sin t$ ($t\in[0,2\pi]$)이므로
미분은 $C'(t) = -r\sin t + ir\cos t = ir(\cos t + i\sin t) = ir\exp(it)$ ($t\in[0,2\pi]$)이다.
두 가지 경우의 적분을 각각 계산하면,

\begin{itemize}
\item[$1^\circ$] $n=-1$일 때,
\begin{align*}
\int_C (z-z_0)^n dz
&= \int_C (z-z_0)^{-1} dz = \int_0^{2\pi} \dfrac1{r\exp(it)}\cdot ir \exp(it)dt \\
&= \int_0^{2\pi} i dt  = 2\pi i.
\end{align*}
\item[$2^\circ$] $n\ne -1$일 때,
\begin{align*}
\int_C (z-z_0)^n dz
&= \int_0^{2\pi} r^n \exp(nit) \cdot ir \exp(it)dt \\
&= \int_0^{2\pi} ir^{n+1} \exp(i(n+1)t)dt \\
&= -r^{n+1} \int_0^{2\pi} \sin((n+1)t)dt + 
ir^{n+1} \int_0^{2\pi} \cos((n+1)t)dt \\
&= 0+0 = 0.
\end{align*}
\end{itemize}
이로써 증명이 완성된다.
\hfill $\square$

우리는 나중에 이 결과가 의미심장한 결과를 갖는다는 것을 보게 될 것이다.
예를 들어, 
중심이 $z_0$, 안쪽 반지름이 $r$, 바깥쪽 반지름이 $R$인 
원환 $\mathbb A := \{ z\in\mathbb C \,:\, r<|z-z_0|<R \}$에 정의된
$f$가 $z$에 대하여 ``정수 지수를 갖는 항으로 된 급수전개''를 가진다고 가정하자 
(그 의미가 무엇이든).
\[
f(z) = \sum_{n\in\mathbb Z} a_n (z-z_0)^n, \quad z\in \mathbb A.
\]
이 (무한) 합의 정확한 의미는 나중에 알아보기로 하고
지금은 단지 유한 합 (유한개의 $a_n$을 제외하고는 모두 $0$되는)이라 생각해도 된다.
그러면  양변에 $(z-z_0)^{-(m+1)}$ ($m\in \mathbb Z$)을 곱하여 
\[
\dfrac{f(z)}{(z-z_0)^{m+1}} = \sum_{n\in\mathbb Z} a_n (z-z_0)^{n-m-1}
\]
을 얻게 되므로
\[
\dfrac1{2\pi i} \int_C \frac{f(z)}{(z-z_0)^{m+1}} dz
= \sum_{n\in\mathbb Z} a_n \int_C (z-z_0)^{n-m-1}dz = a_m.
\]
여기서 우리는 $C$을 따르는 경로적분과 합의 순서를 바꿀 수 있다고 가정했는데,
유한 합의 경우는 다음 절에서 적분의 정의로부터 가능함을 보일 것이다.
무한 합의 경우는 나중에 정확한 의미를 만들어 가겠지만 근본적으로는
제안한 계산 방식대로 작동한다.
궁극적으로는 계수들이 경로적분으로 표현될 수 있다는 것을 의미하며,
우리는 나중에 원환에 정의된 복소해석함수는 항상 이러한 급수 표현을 갖는다는 것을
보일 예정이다.

\begin{salt_exercise} \label{ex-3-5}
$C$를 중심이 $0$이고 반지름이 $1$인 원을 반시계방향으로 도는 경로라고 하자.
$0\le k \le n$에 대하여 다음이 성립함을 보여라.
\[
{n \choose k} = \dfrac1{2\pi i}\int_C \dfrac{(1+z)^n}{z^{k+1}} dz.
\]
\end{salt_exercise}

\section{경로적분의 성질}

이 절에서는 경로적분의 몇가지 유용한 성질을 살펴볼 것이다.
다음 정리는 경로적분의 정의로부터 직접 얻을 수 있다.

\begin{salt_prop} \label{prop-3-1}
복소수 $\mathbb C$의 영역 $D$에 대하여
$\gamma: [a,b] \to D$가 조각적으로 연속인 경로라고 하자.
그러면 다음이 성립한다.
\begin{itemize}
\item[(1)] 연속함수 $f,g : D \to \mathbb C$에 대하여,
\[
\int_\gamma (f+g)(z) dz = \int_\gamma f(z)dz + \int_\gamma g(z)dz.
\]
\item[(2)] 연속함수 $f : D \to \mathbb C$와 상수 $\alpha\in\mathbb C$에 대하여,
\[
\int_\gamma  (\alpha f)(z)dz = \alpha \int_\gamma f(z)dz.
\]
\end{itemize}
\end{salt_prop}
$C(D;\mathbb C)$를
$D$에 정의된 연속인 복소함수의 (점별 연산에 대한) $\mathbb C$상의 벡터공간이라 하자.
\footnote{역주: 점별연산이란 $f$, $g$의 합 $h:=f+g$을 $h(z):=f(z)+g(z)$로 정의함을 뜻한다}
그러면 위 결과는 $D$에 속하는 조각적으로 매끄러운 경로 $\gamma$는
$C(D;\mathbb C)$에서 $\mathbb C$로의 선형변환을 유도함을 의미한다.
\footnote{역주: $T: C(D;\mathbb C) \to \mathbb C$가 선형변환이면
$T(f+g) = T(f)+T(g)$, $T(\alpha f) = T(f)$를 만족한다. }
즉,
\[
f \mapsto \int_\gamma f(z)dz : C(D;\mathbb C) \to \mathbb C.
\]

\begin{salt_exercise} \label{ex-3-6}
명제 \ref{prop-3-1}을 증명하라.
\end{salt_exercise}

{\bf 반대경로:}
영역 $D$에 매끄러운 경로 $\gamma: [a,b] \to D$가 
주어졌을 때, {\bf 반대경로} $-\gamma: [a,b] \to D$는
$(-\gamma)(t) = \gamma(a+b-t)$, $t\in[a,b]$로 정의한다.
그러면 $(-\gamma)(a) = \gamma(b)$, $(-\gamma)(b) = \gamma(a)$이다.
따라서 $-\gamma$는 $\gamma$의 끝점에서 시작하고,
$\gamma$의 시작점에서 끝나는 경로이며,
$\gamma$와 동일한 길을 반대 방향으로 이동한다.
그림 \ref{fig-3-9}를 보라.
\begin{figure}[!h]
\begin{center}
\includegraphics[width=0.5\textwidth]{./SaltChapter/fig-3-9}
\end{center}
\caption{경로 $\gamma$의 반대경로 $-\gamma$}
\label{fig-3-9}
\end{figure}

그런데 왜 반대경로를 $-\gamma$라고 쓸까?
그 이유는 다음과 같다.

\begin{salt_prop} \label{prop-3-2}
$\gamma: [a,b] \to D$가 영역 $D$의 매끄러운 경로이고
$f:D\to\mathbb C$가 연속함수라고 하자. 그러면
\[
\int_{-\gamma} f(z)dz = - \int_\gamma f(z)dz.
\]
\end{salt_prop}

{\bf 증명}

\begin{align*}
\int_{-\gamma} f(z)dz
&= \int_a^b f((-\gamma)(t))\cdot (-\gamma)'(t)dt \\
&= \int_a^b f(\gamma(a+b-t))\cdot (\gamma'(a+b-t))\cdot(-1)dt \\
& \stackrel{(\tau=a+b-t)}=
\int_b^a f(\gamma(\tau))\cdot \gamma'(\tau)d\tau 
= - \int_a^b f(\gamma(\tau))\cdot \gamma'(\tau)d\tau \\
&= - \int_\gamma f(z)dz.
\end{align*}
\hfill $\square$

\begin{salt_exercise} \label{ex-3-7}
$\gamma: [a,b] \to D$가 영역 $D$의 매끄러운 경로일 때,
$-(-\gamma) = \gamma$임을 보여라.
\end{salt_exercise}

{\bf  경로의 결합:}
영역 $D$에 대하여, 두 경로
\begin{align*}
\gamma_1 &: [a_1, b_1] \to D, \\
\gamma_2 &: [a_2, b_2] \to D
\end{align*}
가 다음을 만족한다고 하자.
\[
\gamma_1(b_1) = \gamma_2(a_2)
\]
(그러면 $\gamma_2$는 $\gamma_1$의 끝점에서 시작한다.)
경로의 결합 $\gamma_1+\gamma_2: [a_1, b_1+b_2-a_2] \to D$를
다음과 같이 정의한다.
\[
(\gamma_1+\gamma_2)(t) = \begin{cases}
\gamma_1(t), & a_1\le t\le b_1, \\
\gamma_2(t-b_1+a_2), & b_1 \le t \le b_1+b_2-a_2.
\end{cases}
\]

\begin{figure}[!h]
\begin{center}
\includegraphics[width=0.6\textwidth]{./SaltChapter/fig-3-10}
\end{center}
\caption{두 경로 $\gamma_1$과  $\gamma_2$의 결합 $\gamma_1 + \gamma_2$}
\label{fig-3-10}
\end{figure}

\begin{salt_prop} \label{prop-3-3}
$D$가 복소평면 $\mathbb C$의 영역이고
두 경로 $\gamma_1: [a_1,b_1] \to D$와 $\gamma_2: [a_2,b_2] \to D$가 
$\gamma_1(b_1) = \gamma_2(a_2)$를 만족한다고 하자.
그러면
\[
\int_{\gamma_1+\gamma_2} f(z)dz 
=\int_{\gamma_1} f(z)dz + \int_{\gamma_2} f(z)dz.
\]
\end{salt_prop}

{\bf 증명}
\begin{align*}
\int_{\gamma_1+\gamma_2} f(z)dz 
&= \int_{a_1}^{b_1+b_2-a_2} f((\gamma_1+\gamma_2)(t)) (\gamma_1+\gamma_2)'(t)dt\\
&= \int_{a_1}^{b_1} f((\gamma_1+\gamma_2)(t)) (\gamma_1+\gamma_2)'(t)dt\\
& \quad\quad 
+\int_{b_1}^{b_1+b_2-a_2} f((\gamma_1+\gamma_2)(t)) (\gamma_1+\gamma_2)'(t)dt\\
&= \int_{a_1}^{b_1} f(\gamma_1(t))\gamma_1'(t)dt \\
&\quad\quad 
+ \int_{b_1}^{b_1+b_2-a_2} f(\gamma_2(\tau-b_1+a_2))\gamma_2'(\tau-b_1+a_2)d\tau\\
&= \int_{\gamma_1} f(z)dz + \int_{a_2}^{b_2} f(\gamma_2(s)) \gamma_2'(s)ds 
\ (s=\tau-b_1+a_2) \\
&= \int_{\gamma_1} f(z)dz + \int_{\gamma_2} f(z)dz.
\end{align*}

\begin{salt_exercise} \label{ex-3-8}
$\gamma: [a,b] \to D$가 영역 $D$의 매끄러운 경로이고
$f:D\to\mathbb C$가 연속함수라고 하자. 다음을 증명하라.
\[
\dint_{\gamma+(-\gamma)} f(z)dz =0.
\]
\end{salt_exercise}

{\bf 유용한 판정식:}
이제 경로적분의 크기를 경로에서의 $|f|$의 크기, 경로의 길이의 관점에서
나타낸 부등식을 증명한다. 이 부등식은 향후 필수적인 것으로 입증될 것이다.

\begin{salt_prop} \label{prop-3-4}
\
\begin{itemize}
\item[(1)] $D$가 복소평면 $\mathbb C$의 영역이고,
\item[(2)]  $\gamma : [a,b] \to D$가 조각적으로 매끄러운 경로이고,
\item[(3)] $f:D\to\mathbb C$가 연속함수이면,
\end{itemize}
다음 부등식이 성립한다.
\begin{equation} \label{eq-3-3}
\left| \int_\gamma f(z)dz \right| 
\le \left( \max_{t\in[a,b]} |f(\gamma(t))| \right) 
\cdot (\gamma \text{의 길이}).
\end{equation}
$\gamma$의 길이는 
\[
\int_a^b \sqrt{ (x'(t))^2 + (y'(t))^2} dt
\]
로 주어지며 $x,y: [a,b] \to \mathbb R$는
경로 $\gamma$의 실수부와 허수부를 나타낸다.
그림 \ref{fig-3-11}을 보라.
\begin{figure}[!h]
\begin{center}
\includegraphics[width=0.9\textwidth]{./SaltChapter/fig-3-11}
\end{center}
\caption{경로 $\gamma$의 길이는 국소적인 곡선 길이 $ds$의 합이고,
$ds = \sqrt{(x'(t)dt)^2 + (y'(t)dt)^2} = \sqrt{(x'(t))^2 + (y'(t))^2}dt$이다.}
\label{fig-3-11}
\end{figure}
\end{salt_prop}

{\bf 증명}

우선 곡선 $\varphi : [a,b] \to \mathbb C$에 대하여 다음 부등식을 증명하자.
\[
\left| \int_a^b \varphi(t)dt \right|
\le \int_a^b |\varphi(t)|dt.
\]
이를 위해 $\dint_a^b \varphi(t) dt = r\cdot \exp(i\theta)$로 쓰자.
여기서 $r\ge 0$이고 $\theta \in (-\pi, \pi]$이다.
그러면,
\begin{align*}
\left| \int_a^b \varphi(t)dt \right|
&= r = \exp(-i\theta)\cdot r \cdot \exp(i\theta) \\
&= \exp(-i\theta) \cdot \int_a^b \varphi(t)dt
= \int_a^b \exp(-i\theta) \cdot \varphi(t)dt \\
&= \int_a^b \Re(\exp(-i\theta)\cdot\varphi(t))dt
+ i \int_a^b \Im(\exp(-i\theta)\cdot\varphi(t))dt.
\end{align*}
그런데 좌변은 실수이므로, 우변의 허수부 적분은 $0$이 되어야 한다.
따라서 
\begin{align*}
\left| \int_a^b \varphi(t)dt \right|
&= \int_a^b \Re(\exp(-i\theta)\cdot\varphi(t))dt \\
&\le \int_a^b |\Re(\exp(-i\theta)\cdot\varphi(t))|dt \\
&\le \int_a^b | \exp(-i\theta) \cdot \varphi(t)| dt
= \int_a^b |\varphi(t)| dt. 
\end{align*}
$\varphi(t) := f(\gamma(t))\cdot \gamma'(t)$, $t\in[a,b]$라 두면
\begin{align*}
\left| \int_\gamma f(z)dz \right| 
&= \left| \int_a^b f(\gamma(t)) \gamma'(t) dt \right| \\
&\le \int_a^b |f(\gamma(t)) \gamma'(t)| dt
= \int_a^b |f(\gamma(t))| | \gamma'(t)| dt \\
&\le \left( \max_{t\in[a,b]} |f(\gamma(t))| \right) 
\int_a^b |\gamma'(t)|dt.
\end{align*}
실함수 $x, y$를 써서  $\gamma(t) = x(t) + iy(t)$라 쓰면
\[
\int_a^b |\gamma'(t)|dt 
= \int_a^b  \sqrt{ (x'(t))^2 + (y'(t))^2} dt
= \gamma \text{의 길이}
\]
가 되어 증명이 끝난다.
\hfill $\square$

\begin{salt_exercise} \label{ex-3-9}
$\gamma$가 $0$부터 $1+i$까지의 선분일 때,
적분 
\[
\int_\gamma z^2dz
\] 
의 절대값의 상한을 식 \eqref{eq-3-3}에 주어진 방법으로 계산하라.
또한, 직접 적분을 계산하고 절대값을 구하라.
\end{salt_exercise}

\begin{salt_exercise} \label{ex-3-10}
연습문제 \ref{ex-3-5}의 결과를 이용하여
$\displaystyle{2n \choose n} \le 4^n$을 보여라.
\end{salt_exercise}

\section{경로적분의 기본정리}

실함수에 대한 미적분학의 기본정리를 다시 보자.
\begin{salt_theorem}[미적분학의 기본정리] \label{thm-3-2}
$F:[a,b] \to \mathbb R$가 연속미분가능하고, 
$[a,b]$에서 $F'=:f$라 하면,
\[
\int_a^b f(x)dx = F(b) - F(a).
\]
\end{salt_theorem}

이 정리는 리만 적분의 계산을 용이하게 해주는 중요한 정리이다. 
실제로 함수가 어떤 함수의 도함수인 것을 안다면, 정적분을 쉽게 계산할 수 있다.
예를 들면,
\[
x^2 = \dfrac d{dx}\left( \dfrac{x^3}3\right) \text{을 이용하면, }
\int_a^b x^2 dx = \dfrac{b^3-a^3}3.
\]
유사하게,  함수 $f$가 복소해석함수의 미분이라면 
실함수에 대한 미적분학의 기본정리와 비슷한 다음 정리로부터 
경로적분
\[
\int_\gamma f(z)dz
\]
의 계산이 쉽게 얻어진다.

\begin{salt_theorem}[경로적분에 대한 미적분학의 기본정리] 
\footnote{실해석의 결과와 유사함을 강조하기 위해  정리의 이름에
``기본''이란 용어를 사용하였다. 하지만, 복소해석학에서는
그만큼 ``근본적( fundamental)''이진 않다. 
더 확실하게 근본적인 코시 적분정리를 곧 배우게 될 것이다. }
\label{thm-3-3}
\
\begin{itemize}
\item[(1)] $D$가 복소평면 $\mathbb C$의 영역이고,
\item[(2)] $\gamma : [a,b] \to D$가 조각적으로 매끄러운 경로이고,
\item[(3)] $f:D\to\mathbb C$가 $D$에서 연속함수이고,
\item[(4)] $F:D\to \mathbb C$가 $D$에서 $F'=f$를 만족하는 복소해석함수이면,
\end{itemize}
\[
\int_\gamma f(z)dz = F(\gamma(b)) - F(\gamma(a)).
\]
\end{salt_theorem}

이 정리가 어떤 도움을 줄까?
이제 우리는 어떤 경로적분들은 매우 쉽게 계산할 수 있다
(보통의 미적분에서와 마찬가지로).
아래 예제를 보자.

\begin{salt_example}\label{example-3-5}
$z\in\mathbb C$에 대하여
$\dfrac d{dz}\left( \dfrac{z^2}2\right)=z$이므로,
$0$부터 $1+i$까지의 임의의 경로 $\gamma$에 대한 적분은
\[
\int_\gamma z\, dz = \dfrac{(1+i)^2}2 - \dfrac{0^2}2 
= \dfrac{1+2i+i^2}2 = \dfrac{1+2i-1}2 = i
\]
가 되어 예제 \ref{example-3-4}와 같은 결과를 얻는다.
\hfill $\diamondsuit$
\end{salt_example}

앞의 예제에서 살펴본 바와 같이 
$D$에서 $f$가 ``부정적분'' 또는 ``원시함수''라 불리는 $F$를 가지면
\[
\int_\gamma f(z)dz = F(w) - F(z)
\]
은 $z$와 $w$를 잇는 경로 $\gamma$에 무관하다.

\begin{salt_example}\label{example-3-6}
모든 $\mathbb C$에 대하여 
$F'(z) = \bar z$를 만족하는 함수 $F:\mathbb C \to \mathbb C$는 없다.
실제로 예제 \ref{example-3-2}와 \ref{example-3-3}의 계산은
$0$부터 $1+i$까지의 적분이 경로에 의존적임을 보여준다.
\hfill $\diamondsuit$
\end{salt_example}

{\bf 증명} (정리 \ref{thm-3-3})

$z = x+iy\in D$ ($x,y\in \mathbb R$)에 대하여
실함수 $U$, $V$, $u$, $v$를 다음과 같이 정의하자.
\begin{align*}
F(x+iy) &= U(x,y) + iV(x,y), \\
f(x+iy) &= u(x,y) + iv(x,y).
\end{align*}
또한, $\gamma(t) = x(t) + iy(t)$ ($t\in[a,b]$)라고 하자.
여기서 $x$, $y$는 실함수이다.
그러면, 코시-리만 방정식에 의해
\begin{align*}
u(x,y) + iv(x,y) 
&= f(x+iy) = F'(x+iy) \\
&= \dfrac{\partial U}{\partial x}(x,y) + i \dfrac{\partial V}{\partial x}(x,y)
= \dfrac{\partial V}{\partial y}(x,y) - i \dfrac{\partial U}{\partial y}(x,y).
\end{align*}
위 식에 연쇄법칙을 적용하면,
\begin{align*}
\dfrac d{dt} U(x(t), y(t))
&= \dfrac{\partial U}{\partial x}(x(t),y(t))\cdot x'(t)
 + \dfrac{\partial U}{\partial y}(x(t),y(t))\cdot y'(t) \\
&= u(x(t),y(t))\cdot x'(t) - v(x(t),y(t))\cdot y'(t).
\end{align*}
비슷한 방법으로,
\begin{align*}
\dfrac d{dt} V(x(t), y(t))
&= \dfrac{\partial V}{\partial x}(x(t),y(t))\cdot x'(t)
 + \dfrac{\partial V}{\partial y}(x(t),y(t))\cdot y'(t) \\
&= v(x(t),y(t))\cdot x'(t) + u(x(t),y(t))\cdot y'(t).
\end{align*}
따라서,
\begin{align*}
\int_\gamma f(z)dz
&= \int_a^b f(\gamma(t))\gamma'(t)dt \\
&= \int_a^b \left( u(x(t), y(t)) + iv(x(t),y(t)) \right) (x'(t)+iy'(t))dt \\
&= \int_a^b \dfrac d{dt} U(x(t), y(t))dt + i \int_a^b \dfrac d{dt} V(x(t), y(t))dt \\
&= U(x(b), y(b)) - U(x(a), y(a)) + 
i\left( V(x(b), y(b)) - V(x(a), y(a)) \right) \\
&= F(\gamma(b)) - F(\gamma(a)).
\end{align*}
이로써 증명이 끝난다. \hfill $\square$







